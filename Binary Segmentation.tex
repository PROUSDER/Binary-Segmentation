\documentclass[a4paper, 12pt, notitlepage]{report}

\usepackage{amsmath, amssymb, amsfonts, amsthm}
%\usepackage{cite}
\usepackage{graphicx}

\title{Binary Segmentation}
\author{Louis Forbes Wright}
\date{\today}

\newtheorem{thm}{Theorem}[section]
%\newlemma{lem}{Lemma}[section]

\begin{document}
\maketitle

%citation example. \cite{Venk}


%We are interested in the problem of seperating the signal from noise in a given dataset. How do we do this?

%Section 1 is an introduction the problem 
%Section 2 defines and expands some useful definitions we will be using
%Section 3 focuses on the theorem of Binary Segmentation and its associated lemmas
%Section 4  looks at an algorithm based on the theorem and from monte carlo simulations how consistent it is 
%Section 5 applies the theorem to some actual data collected from ...

%\bibliography{mybib} { }
%\bibliographystyle{plain}


%%%%%%%%%% PRELIMINARY MATERIAL %%%%%%%%%%
%\maketitle
\begin{center}
A project of 20 credit points at level 4. % change this
\\[12pt]
Supervised by Dr.\ Haeron Cho. 
\end{center}
\thispagestyle{empty}
\newpage
\section*{Acknowledgement of Sources} % this must be included in undergradate projects
For all ideas taken from other sources (books, articles, internet), the source of the ideas is mentioned in the main text and fully referenced at the end of the report.

All material which is quoted essentially word-for-word from other sources is given in quotation marks and referenced.

Pictures and diagrams copied from the internet or other sources are labelled with a reference to the web page or book, article etc.
\\[12pt]
Signed \dotfill Date \dotfill

\newpage
\section*{Abstract}
We are interested in the problem of seperating the signal from noise in a given dataset. How do we do this?
we are interested in the consistent estimation of the number and location of change-points and means in the a fixed sample.

Section 1 is an introduction the problem 
Section 2 defines and expands some useful definitions we will be using
Section 3 focuses on the reinterpreting the theorem of Binary Segmentation and its associated lemmas
Section 4  looks at an algorithm based on the theorem and from monte carlo simulations how consistent it is 
Section 5 applies the theorem to some actual data collected from ...

Many thanks to ... who is the inspiration and source of many of the results within this work %\cite{Venk}.


\tableofcontents 

%%%%%%%%%% MAIN TEXT STARTS HERE %%%%%%%%%%

%%%%%%%%%% SAMPLE CHAPTER %%%%%%%%%%
\chapter{Introduction}
%
"A change-point is a point after which the stochastic behaviour of a process changes" 
 Let .

Rest of the text introducing this chapter.

\chapter{Important definitions}
%

\chapter{Summary of the theorem}
%
We can now begin our examination of the theorem.
%\begin{lem}

\begin{equation}

Let W be as defined in chapter 2. Then $ \displaystyle P(B_n) \to 1 as n \to \infty $, where

\end{equation}

%\end{lem}
%
%\section{}
%
%\section{}
%
%\section{}
%
%\section{}
%
%\section{}
%
%\section{}
%

\chapter{Consistency estimation}
%

\chapter{Real world Application}
%%%%%%%%%% INFORMATION %%%%%%%%%%
\section{A Note About References}
%
The Third and Fourth Year Handbook, and the Third and Fourth Year Project Handbook, have some clear guidelines about plagiarism and referencing.
You should consult your project supervisor about the correct format for handling references.

This document uses the `in-text' or Harvard system of referencing, which is a good default format.
This requires both in-text citations and a list of references at the end of the document.
The project templates have an example of a list of references.

Within the text you must cite the authors surname(s) and the date of publication.
When referring to a specific idea, or a direct quote, you must also give the page number.
If there are two authors, use `and' and if there are more, use `et al.'\ and give all the authors names at the end.

There are two styles of citation, implicit and explicit.
Both are equally acceptable and it is also acceptable to mix and match.

\subsection{Examples of implicit in-text citations}
%
The sum of convex functions is itself convex (Blacke, 1985) and therefore any minimiser of this objective function will be a global minimiser (Greene and Whit, 1995, p.123). It is possible to exploit this fact (Browne et al., 2005) to enhance the optimisation algorithm.

\subsection{Examples of explicit in-text citations}
%
Blacke (1985) first proved that a sum of convex functions is convex. Any minimiser of this objective function will be a global minimiser, a fact shown by Greene and Whit (1995, p.123) and exploited by Browne et al.\ (2005) to enhance the optimisation algorithm.

%%%%%%%%%% APPENDIX %%%%%%%%%%
\appendix
\chapter{An Appendix may not be necessary}
%
Text introducing the/this appendix.

\section{Appendix section}
%
Text of this section.

Subsections and further divisions can also be used in appendices.

%%%%%%%%%% BIBLIOGRAPHY %%%%%%%%%%
%\bibliographystyle{plain}
%\bibliography{mybib} { }

%


\end{document}



